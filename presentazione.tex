% presentazione.tex

\documentclass{beamer}


\usetheme{Warsaw} % tema della presentazione
\usecolortheme{seahorse} % per grafica della presentazione 
\usepackage{listings} % per caricare il codice (R) dall'esterno

\setbeamertemplate{blocks}[rounded][shadow = true]
% bordo arrotondato e una certa ombreggiatura

\title{My first presentation}
\author{Chiara Salvatori}
\institute{Alma Mater Studiorum - Univeristà di Bologna\\
\bigskip
\includegraphics[width = 0.5\textwidth]{logo_unibo.png}
}
% da qui possiamo iniziare con il documento...

\AtBeginSection[] % Do nothing for \section*
{   
\begin{frame}
\frametitle{Outline}
\tableofcontents[currentsection,currentsubsection,currentsubsubsection]
\end{frame}
} % per fare il sommario all'inizio e alla fine delle sezioni

\begin{document}
\maketitle

\section{Introduction}
    \begin{frame}
        \frametitle{This is my first slide!}
        \scriptsize{In this study I will face remote sensing problems...}
        % text sizes: https://it.overleaf.com/learn/latex/Font_sizes,_families,_and_styles
    \end{frame}

%slide elenchi puntati
\begin{frame} % nuova diapositiva
\frametitle{Itemizing - Elenchi puntati} %titolo della diapositiva
\begin{itemize} % inizio dell'elenco puntato
    \item<1-> Remote sensing is a powerful tool 
    \item<2-> It can be used for diversity estimate  % tutti gli elementi dell'elenco
    \item<3-> It can be used in ecological informatics
\end{itemize}
\end{frame}

\section{Methods}
\begin{frame}
        \frametitle{Inseriamo una formula}
        \begin{equation}
            % formula della dev.standard
            I = \frac{\sqrt[3]{\frac{\displaystyle\sum_{i = 1}^{N} S}{N}}}{\alpha_{\gamma_{\sqrt{\beta}}}}
            % https://en.wikibooks.org/wiki/LaTeX/Mathematics#Brackets,_braces_and_delimiters
        \end{equation}
\end{frame}

\section{Coding}
\begin{frame}
\frametitle{Codice R}
\lstinputlisting[language = R]{code_r.txt}
\end{frame}

\section{Immages}
\begin{frame}
\frametitle{Immagine}
\centering
\includegraphics[width = 0.6\textwidth]{swamp.jpg}
\\ % andare a capo
\bigskip 
Questi sono i risultati della mia analisi
\end{frame}

\begin{frame}{Quattro immagini insieme}
\centering
\includegraphics[width = 0.3\textwidth]{swamp.jpg}
\includegraphics[width = 0.3\textwidth]{swamp.jpg} 
\\
\includegraphics[width = 0.3\textwidth]{swamp.jpg}
\includegraphics[width = 0.3\textwidth]{swamp.jpg}
\end{frame}


\end{document}
