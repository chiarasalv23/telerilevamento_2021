% Prova LaTex!

% \"funzioni{"argomenti"}
\documentclass[a4paper, 12pt]{article}
\usepackage[utf8]{inputenc}
\usepackage{hyperref} %puntatori?
\usepackage{graphicx} % pacchetto x parti grafiche
\usepackage{lineno}
% \linenumbers

% Titolo
\title{This is my first LaTex document!}
\author{Chiara Salvatori $^1$, Simone Plm $^2$} % se ho tanti autori serve 

\begin{document}
\maketitle

$^1$ Alma Mater Studiorum University of Bologna
$^2$ SNASA
% abstract della tesi
\begin{abstract}
This is my cool abstract

My thesis is dealing with something...

Congluding: I'M COOL!
\end{abstract}
% se voglio andare a capo tra i due lascio una riga
\section{Introduction}\label{section:intro} % sezione per capitoli!!!
% \section*{Introduction}senza 1 davanti
% qui possiamo iniziare a scrivere la tesi di laurehahahaha
Signoraaaaaaaa...

...i limoniiiiiiii
\section{Method}
Methods della tesi. Come ho detto nella sezione \ref{section:intro}
\subsection{Study area}
My study area is the land of misanthropy and malaria 
\subsection{Fields...}
write something... (Figure \ref{fig:fronk}).
\begin{figure}
\include{rana.jpg}
    \centering
    \includegraphics[width=1\textwidth]{rana.jpg}
    \caption{fronk boi w/ cramberries}
    \label{fig:fronk}
\end{figure}
\end{document}
